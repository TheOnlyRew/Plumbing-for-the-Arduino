\DeclareGraphicsRule{.tif}{png}{.png}{`convert #1 `dirname #1`/`basename #1 .tif`.png}

\newcommand{\plumbing}{Plumbing\xspace}

\newcommand{\GOALS}{\section{Goals}}
\newcommand{\CODE}{\section{Code}}
\newcommand{\PATTERNS}{\section{Patterns}}
\newcommand{\BREAKAGE}{\section{Breakage}}
\newcommand{\EXPLORATIONS}{\subsection{Explorations}}

\newcommand{\PROCedure}{{\ttfamily\bfseries PROC}edure\xspace}
\newcommand{\PROC}{{\ttfamily\bfseries PROC}\xspace}
\newcommand{\PARallel}{{\ttfamily\bfseries PAR}allel\xspace}
\newcommand{\PAR}{{\ttfamily\bfseries PAR}\xspace}


\newcommand{\code}{\ttfamily}
\newcommand{\constant}{\ttfamily}
\newcommand{\procname}{\ttfamily}
\newcommand{\keyword}{\ttfamily}

\newcommand{\XXX}{{\em TO BE WRITTEN}\xspace}

\newcommand{\HIGH}{{\code HIGH}\xspace}
\newcommand{\LOW}{{\code LOW}\xspace}

\newcommand{\webnote}[2]{\footnote{See \url{#1}\xspace for more information about #2.}}
\newcommand{\ohm}{$\tcohm$\xspace}

 \definecolor{Brown}{cmyk}{0,0.81,1,0.60}
 \definecolor{OliveGreen}{cmyk}{0.64,0,0.95,0.40}
 \definecolor{CadetBlue}{cmyk}{0.62,0.57,0.23,0}
 
% Definition for typesetting occam source code
\lstdefinelanguage{occam} 
{
	keywords={PROC,PAR,SEQ,WHILE,ALT,CHAN,SIGNAL,LEVEL},
	sensitive=true,
	comment=[l]{---},
	string=[b]",
	frame=ltrb,
	framesep=5pt,
  keywordstyle=\color{OliveGreen}\bfseries,
  identifierstyle=\color{CadetBlue}\bfseries, 
  commentstyle=\color{Brown},
  % stringstyle=\ttfamily,
  showstringspaces=true
}
% Layout of source code in general
\lstset{xleftmargin=14pt,basicstyle=\small\fontfamily{pcr}\selectfont, captionpos=b,numbers=left,numbersep=10pt,language=occam}

\renewcommand\lstlistingname{Figure}

\newcommand{\mypi}{$\pi$}
\newcommand{\myPI}{\large$\pi$}
\newcommand{\occam}{{\fontfamily{phv}\selectfont occam-\myPI}\xspace}
\newcommand{\occamone}{{\fontfamily{phv}\selectfont occam1}}
\newcommand{\loccam}{{\fontfamily{phv}\selectfont loccam}}
%\newcommand{\kroc}{{\fontfamily{phv}\selectfont soccam}}
\def\kroc{KR{\sffamily o}C\xspace} 
\newcommand{\tvm}{Transterpreter\xspace}
\newcommand{\Tvm}{Transterpreter\xspace}


\newcommand{\tvmurl}{\url{www.transterpreter.org}\xspace}
\newcommand{\tvmcite}{\cite{transterpreter}}

\newcommand{\strong}{\bfseries}

